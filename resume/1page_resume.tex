\documentclass[margin,line, 9pt]{res}

\usepackage{hyperref}
\usepackage{amsmath}
\usepackage{textcomp}
\usepackage{color}
\usepackage{lettrine}
% \usepackage[left=0.75in, right=0.75in, top=0.75in,bottom=0.75in, margin=0.75in]{geometry}
\oddsidemargin = -.5in
\evensidemargin = -.5in
\topmargin = -0.5in
\textheight = 9.75in
\textwidth = 6.0in
\itemsep=0in
\parsep=0in
% if using pdflatex:
%\setlength{\pdfpagewidth}{\paperwidth}
%\setlength{\pdfpageheight}{\paperheight} 
% 
\newenvironment{list1}{
  \begin{list}{\ding{113}}{%
      \setlength{\itemsep}{0in}
      \setlength{\parsep}{0in} \setlength{\parskip}{0in}
      \setlength{\topsep}{0in} \setlength{\partopsep}{0in} 
      \setlength{\leftmargin}{0.17in}}}{\end{list}}
\newenvironment{list2}{
  \begin{list}{$\bullet$}{%
      \setlength{\itemsep}{0.03in}
      \setlength{\parsep}{0in} \setlength{\parskip}{0in}
      \setlength{\topsep}{0in} \setlength{\partopsep}{0in} 
      \setlength{\leftmargin}{0.2in}}}{\end{list}}
% 
\tolerance=1
\emergencystretch=\maxdimen
\hyphenpenalty=10000
\hbadness=10000
% 
\pagenumbering{gobble}
% 
% 
\begin{document}
% 
\textbf{\huge{Pranav Sankhe}}

\section{\sc Contact Information}
% \vspace{.05in}
\begin{tabular}{@{}p{2.9in}p{.5in}p{3in}}
Department of Electrical Engineering & \multicolumn{1}{r}{\it Phone:}  &(+91) 902 920 4916 \\            
Indian Institute of Technology Bombay &\multicolumn{1}{r}{\it E--Mail:}& \href{mailto:pranavsankhe40@gmail.com}{\textcolor{blue}{pranavsankhe40@gmail.com}} \\ 
\end{tabular}

% \begin{resume}
% \section{\sc Contact Information}
% \vspace{.05in}
% \begin{tabular}{@{}p{2.9in}p{.5in}p{3in}}
% Department of Electrical Engineering & \multicolumn{1}{r}{\it Phone:}  &(+91) 902 920 4916 \\            
% Indian Institute of Technology Bombay &\multicolumn{1}{r}{\it E--Mail:}& \href{mailto:pranav_sankhe@iitb.ac.in}{\textcolor{blue}{pranav\_sankhe@iitb.ac.in}} \\ 
% \#132, Hostel 07, IIT Bombay & & \href{mailto:pranavsankhe40@gmail.com}{\textcolor{blue}{pranavsankhe40@gmail.com}} \\ 
% Powai, Mumbai, India 400 076 & \multicolumn{1}{r}{\it Webpage:} &\href{https://sabsathai.github.io}{\textcolor{blue}{https://sabsathai.github.io}} \\     
% \end{tabular}
% 
% 
% 
% \section{\sc Research Interests}
% I am passionate about Time Series Processing (Video Processing, Audio Processing, Wireless Communication, Music Transcription), Machine Learning (Deep Learning Models, Bayesian statistics), Computer Vision and Computational Neuroscience. I am interested in fundamental questions which surround us.
% 
% 
% 
% \section{\sc Education}
% {\bf \href{http://www.iitb.ac.in/}{\textcolor{blue}{Indian Institute of Technology Bombay}}}, Mumbai, India \hfill {\it July 2015 -- Present} \\
% \vspace*{-.13in}
% \begin{list1}
% \item[] Fourth Year, Dual Degree (Bachelor \& Master of Technology), Department of \href{http://www.ee.iitb.ac.in/}{\textcolor{blue}{Electrical Engineering}}
% \item[] Specialization: {\em Signal Processing \& Communication}
% \end{list1}
%
\section{\sc Publications  \\ \& Patents}
\begin{list2}
\item Pranav Sankhe, {\em An Information Theoretical Approach Towards the Reconstruction of Tempo from EEG Responses}. Accepted at \textbf{CogMIR 2019}. \emph{Awarded the Best Paper Award}
%
\item Pranav Sankhe, Animesh Kumar {\em Cortical Representations of Auditory Perception using Graph Independent Component Analysis on EEG}. Accepted at \textbf{AESOP 2019}.
%
\item Sankhe, P., Azim, S., Goyal, S., Et al., {\em Indoor Positioning System using LSTMs over WLAN Network}. Accepted at \textbf{IEEE WPNC 2019.}
%
\item Agrim Gupta, Pranav Sankhe, Et al., {\em Predictive Quantization for MIMO-OFDM SVD Precoders using Reservoir Computing Framework}. Submitted to \textbf{IEEE Globalcom 2019.}
% 
\item {Filed a \textbf{patent},\em “Indoor Positioning System using LSTMs over WLAN Network”}, December 2018, \textbf{Indian Patent Office}, Mumbai.
%
%
\end{list2}
%
% \section{\sc Patents}
% \begin{list2}
% \item {Filed a patent,\em “Indoor Positioning System using LSTMs over WLAN Network”}, December 2018, \textbf{Indian Patent Office}, Mumbai.
% %
% % \item \textbf{Patent:} “Predictive Quantization for MIMO-OFDM SVD Precoders using Reservoir Computing Framework”, June 2019, \href{http://www.ipindia.nic.in}{\textcolor{blue} {Indian Patent Office}}, Mumbai.
% \end{list2}
%
%
\vspace*{-.13in}
\section{\sc Internships}
%
{\bf Honda Research Institute, Saitama, Japan} \hfill {\it May'18 -- July'18} \\
\emph{Sign Language Translation using Deep LSTM \& 3D ResNet Networks} \\
% {\em Guide: Dr. Brock Hieke}  \\
% \vspace*{-.13in}
% \vspace*{-.13in}
We worked on the translation of the Sign Language to natural language using Neural Sequence to Sequence Models. We computed the optical flow to extract motion sequences and mapped them to the corresponding text sequences. Our architecture consisted of a decoder and an encoder. The encoder consisted of 3D Convolutional Neural Network with recurrent connection followed by Bidirectional LSTMs for temporal modeling. The decoder model used was a standard decoder of sequence to sequence networks consisting of Bidirectional LSTMs.
% 
\vspace{-.35in} \\
\section{\sc Research Projects}
{\bf Information Theory Approach for Music Reconstrution} \\
{\em Guide: Prof. Prasanna Chaporkar, Electrical Engineering, IIT Bombay} \hfill {\it Dec'18 - May'19}\\
% \vspace*{-.13in}
We developed an information theoretical model to analyze the reconstruction of the tempo of music stimuli from EEG responses. We use mutual information (MI) to access the amount of information transfer from the music stimulus to the EEG response. The obtained MI value establishes a bound over the number of distinguishable stimuli and the on the rate of change of tempo within a stimulus. This preliminary research establishes that using information theory, we can comment over the nature of input stimulus and establish bounds within which the reconstruction of the music stimulus is possible. 
\vspace{.1in} \\
% 
{\bf Cortical representations of Auditory Perception using Graph ICA} \\
{\em Course Project Guide: Prof. Animesh Kumar, IIT Bombay} \hfill {\it Aug'18 - June'19}\\
Recent studies indicate that the neurons involved in a cognitive task aren’t locally limited but span out to multiple regions of the human brain. We obtain network components and their locations for the task of listening to music. To identify these intrinsic cognitive subnetworks corresponding to music perception, we propose to decompose the whole brain graph-network into multiple subnetworks using Graph Independent Component Analysis. We observed that the location of the computed electrodes match with the neuroscientific literature. The weight of these subnetworks increases with the increase in the tempo of the music recording. The results suggest that whole-brain networks can be decomposed into independent subnetworks and analyze cognitive responses to music stimulus.
\vspace{.1in} \\ 
% 
% \vspace*{-.1in}
{\bf Corrupted Speech Processing using Perceptive Models and Spiking Neural Networks} \\
{\em Course Project Guide: Prof. Udayan Ganguly, IIT Bombay} \hfill {\it Aug'18 - Dec'18}\\
We implemented a Source Separation system using auditory scene analysis to seperate speech from the background noise. We used a $2$ layered Spiking Neural Network to generate a mask which essentially removed the ambient noise and gives filtered speech signal.
% 
\vspace{.1in} \\
{\bf Indoor Positioning System using LSTMs over WLAN Network} \\
We designed and developed a self-adaptive WiFi-based indoor distance estimation system using LSTMs. The system is novel in its method of estimating with high accuracy the distance of an object by overcoming possible causes of channel variations and is self-adaptive to the changing environmental conditions. We show that the LSTM based model achieves an accuracy of 5.85 cm with a confidence interval of $93\%$ on the scale of (8.46 m × 6.98 m). \\ 
\end{document}